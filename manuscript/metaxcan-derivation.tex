% Template for PLoS
% Version 1.0 January 2009
%
% To compile to pdf, run:
% latex plos.template
% bibtex plos.template
% latex plos.template
% latex plos.template
% dvipdf plos.template

\documentclass[10pt]{article}

% amsmath package, useful for mathematical formulas
\usepackage{amsmath}
% amssymb package, useful for mathematical symbols
\usepackage{amssymb}

\usepackage{soul}

% graphicx package, useful for including eps and pdf graphics
% include graphics with the command \includegraphics
\usepackage{graphicx}

% cite package, to clean up citations in the main text. Do not remove.
\usepackage{cite}

\usepackage{color} 

\usepackage{hyperref} 

% Use doublespacing - comment out for single spacing
\usepackage{setspace} 
\doublespacing


% Text layout
\topmargin 0.0cm
\oddsidemargin 0.5cm
\evensidemargin 0.5cm
\textwidth 16cm 
\textheight 21cm

% Bold the 'Figure #' in the caption and separate it with a period
% Captions will be left justified
\usepackage[labelfont=bf,labelsep=period,justification=raggedright]{caption}

% Use the PLoS provided bibtex style
%\bibliographystyle{abbrv}
\bibliographystyle{plos2015}

% Remove brackets from numbering in List of References
\makeatletter
\renewcommand{\@biblabel}[1]{\quad#1.}
\makeatother


% Leave date blank
\date{}

\pagestyle{myheadings}
%% ** EDIT HERE **

% bold in tables
\usepackage{array}
\newcolumntype{$}{>{\global\let\currentrowstyle\relax}}
\newcolumntype{^}{>{\currentrowstyle}}
\newcommand{\rowstyle}[1]{\gdef\currentrowstyle{#1}%
  #1\ignorespaces
}

%% ** EDIT HERE **
%% PLEASE INCLUDE ALL MACROS BELOW

%% END MACROS SECTION

\begin{document}

%% Title must be 150 characters or less
\begin{flushleft}
{\Large
\textbf{MetaXcan: Summary Statistics Based Gene-Level Association Method Infers Accurate PrediXcan Results}
}
% Insert Author names, affiliations and corresponding author email.

Alvaro Barbeira$^{1}$, 
Kaanan P. Shah$^{2}$, 
Jason M. Torres$^3$,
Heather E Wheeler$^4$,
Eric S. Torstenson$^{5}$,
Todd Edwards$^{5}$,
Tzintzuni Garcia$^{6}$,
%GTEx Consortium,
Graeme I Bell$^{7}$,
Dan Nicolae$^{1}$,
Nancy J Cox$^{5}$,
Hae Kyung Im$^{2,\ast}$
\\
\bf{1} Department of Physics, Instituto Tecnologico de Buenos Aires, CABA, Argentina 
\\
\bf{2} Section of Genetic Medicine, The University of Chicago, Chicago, IL, USA
\\
\bf{3} Committee on Molecular Metabolism and Nutrition, The University of Chicago, Chicago, IL, USA
\\
\bf{4} Departments of Biology and Computer Science, Loyola University Chicago, Chicago, IL, USA
\\
\bf{5} Vanderbilt Genetic Institute, Vanderbilt University, Nashville, TN, USA
\\
\bf{6} Center for Research Informatics, The University of Chicago, IL, USA
\\
\bf{7} Section of Endocrinology, The University of Chicago, Chicago, IL, USA
\\
$\ast$ E-mail: Corresponding haky@uchicago.edu
\end{flushleft}

%% Please keep the abstract between 250 and 300 words
\section*{Abstract}

To gain biological insight into the discoveries made by GWAS and meta-analysis studies, effective integration of functional data generated by large-scale efforts such as the GTEx Project is needed. PrediXcan is a gene-level approach that addresses this need by estimating the genetically determined component of gene expression. These predicted expression traits can then be tested for association with phenotype in order to test for mediating role of gene expression levels. Furthermore, due to the polygenic nature of many complex traits, efforts to aggregate multiple GWAS studies and conduct meta-analyses have successfully increased our ability to identify variants of small effect sizes. To take advantage of the results generated by these efforts and to avoid the problems associated with accessing and handling individual-level data (e.g. consent limitations, large computational/storage costs) we have developed an extension of PrediXcan. The new method, MetaXcan, infers the results of PrediXcan using only summary statistics from large-scale GWAS or meta-analyses. Here we show that the concordance between PrediXcan and MetaXcan is excellent when the right reference population is used ($R^2 > 0.95$) and robust to population mismatches ($R^2 > 0.85$). We provide open source local and web-based software for easy implementation through \url{https://github.com/hakyimlab/MetaXcan}.

%In sum. MetaXcan is a scalable, accurate and efficient gene-level association test well suited for application to ever increasing sample sizes.


% Please keep the Author Summary between 150 and 200 words
% Use first person. PLoS ONE authors please skip this step. 
% Author Summary not valid for PLoS ONE submissions.   

%\section*{Author Summary}
%
\section*{Introduction}
%
Over the last decade, GWAS have been successful in identifying genetic loci that robustly associate with multiple complex traits. However, the mechanistic understanding of these discoveries is still limited, hampering the translation of this knowledge into actionable targets. Studies of enrichment of expression quantitative trait loci (eQTLs) among trait-associated variants \cite{Nica2010,Nicolae2010} show the importance of gene expression regulation. Direct quantification of the contribution of different functional classes of genetic variants showed that 80\% of phenotype variability (in 12 diseases) can be attributed to DNAase I hypersensitivity sites, further highlighting the importance of transcript regulation in determining phenotypes \cite{Gusev2014}.

Many transcriptome studies have been conducted where genotype and expression levels are assayed for a large number of individuals \cite{Battle2014, Lappalainen2013, Zhang2015,Stranger2012}. The most comprehensive transcriptome dataset, in terms of tissues covered, is the GTEx Project, a large-scale effort where DNA and RNA are collected from multiple tissue samples from nearly 1000 deceased individuals  and sequenced to high coverage \cite{TheGTExConsortium2013}. This remarkable resource provides a comprehensive cross-tissue survey of the functional consequences of genetic variation at the transcript level.
% * <kaananshah@gmail.com> 2015-11-10T19:33:21.089Z:
%
% > reference transcriptome
%
% Do you call these "reference transcriptome" because they are generated on control individuals or because we use them as a reference for predixcan? might be worth clarifying since that phase isnt as standard as "reference haplotypes" or somehting like that for imputation
%
% ^ <kaananshah@gmail.com> 2015-11-10T19:36:00.879Z.
% 
% Switched to ``measured transcriptome'' to avoid confusion with Metaxcan Reference Population - AB
 
To integrate knowledge generated from these large-scale transcriptome studies and shed light on disease biology, we developed PrediXcan \cite{Gamazon2015}, a gene-level association approach that tests the mediating effects of gene expression levels on phenotypes. This is implemented on GWAS/sequencing studies (i.e. studies with genome-wide interrogation of DNA variation and phenotypes) where transcriptome levels are imputed with models trained in measured transcriptome datasets (e.g. GTEx). These predicted expression levels are then correlated with the phenotype and provides the basis for a gene-level association test that addresses some of the key limitations of GWAS \cite{Gamazon2015}.

Other groups have also proposed methods based on similar ideas \cite{Gusev2016}. Comparison with our method will be discussed.

On the other hand, meta-analysis efforts that aggregate results from multiple GWAS studies have been able to identify an increasing number of phenotype associations that were not detected with smaller sample sizes. In order to harness the power of these increased sample sizes while keeping the computational burden manageable, we have extended the PrediXcan method so that only summary statistics from meta-analysis studies are needed rather than individual level genotype and phenotype data. 

We will show here that our new method, termed MetaXcan, is a fast, accurate, and efficient way to scale up implementation of PrediXcan and take advantage of the large sample sizes made available through meta-analysis of GWAS.\\

\section*{Results}

We have derived an analytic expression that allows us to compute the outcome of PrediXcan using only summary statistics from genetic association studies. Details of the derivation are shown in the Methods section. 
In Figure \ref{fig:MetaXcan-PrediXcan-GWAS}, we illustrate the mechanics of MetaXcan in relation to traditional GWAS and our recently published PrediXcan method.

For both GWAS and PrediXcan, the input is the genotype matrix and phenotype vector. GWAS computes the regression coefficient of the phenotype on each marker in the genotype matrix and generates SNP-level results. PrediXcan starts by estimating the genetically-regulated component of the transcriptome (using weights from the publicly available PredictDB database) and then computes regression coefficients of the phenotype on each predicted gene expression level generating gene-level results. MetaXcan, on the other hand,  can be viewed as a shortcut that uses the output from a GWAS study to generate the output from PrediXcan. Since MetaXcan only depends summary statistics, it can effectively take advantage of large-scale meta analysis results, avoiding the computational and regulatory burden of handling large amounts of protected individual level data.
% * <kaananshah@gmail.com> 2015-11-10T19:39:42.284Z:
%
% > Y
%
% What is Y? 
%
% ^ <kaananshah@gmail.com> 2015-11-10T19:39:50.735Z.
%
% ``Y'' is measured phenotype. added statement in figure caption - AB

\begin{figure}
\includegraphics[width=\textwidth]{plots/Fig1-MetaXcan-PrediXcan-GWAS.png}
\caption{This figure illustrates the MetaXcan method in relationship to GWAS and PrediXcan. Both GWAS and PrediXcan take genotype and phenotype data as input. GWAS computes the regression coefficients of $Y\sim X_l$ using the model $Y=X_l b + \epsilon$, where $Y$ is the phenotype and $X_l$ the individual dosage. The output is the table of SNP-level results. PrediXcan, in contrast, starts first by predicting/imputing the transcriptome. Then it calculates the regression coefficients of the phenotype $Y$ on each gene's predicted expression $T_g$. The output is a table of gene-level results. MetaXcan computes the gene-level association results using directly the output from GWAS.} % should the Gene-level Results box include Z-score column for completeness? -HEW
\label{fig:MetaXcan-PrediXcan-GWAS}
\end{figure}

\subsubsection*{MetaXcan formula}

Figure \ref{fig:metaxcan-formula} shows the main analytic expression used by MetaXcan for the Z-score (effect size divided by its standard error) of the association between predicted gene expression and the phenotype. The input variables are the weights used to predict the expression of a given gene $w_{lg}$, the variance and covariances of the markers included in the prediction of the expression level of the gene, and the GWAS coefficient for each marker. The last factor in the formula can be computed exactly in principle, but we would need some additional information that is unavailable in typical GWAS output. Fortunately, we have found that this factor is very close to 1 and dropping it from the formula does not affect the accuracy of the results.

% Describe what additional information you would need? -HEW
% That would statistics figured out from the used genotype for the GWAS, which is usually inaccessible. Rephrased it to make it explicit. - AB
% - I got lost. Where is this shown, exactly? - AB

\begin{figure}
\begin{center}
\includegraphics[width=0.6\textwidth]{plots/Fig2-MetaXcan-Formula.png}
\caption{MetaXcan formula. This plot shows the formula to infer PrediXcan gene-level association results using summary statistics. The different sets involved in input data are shown. The study set is where the regression coefficient between the phenotype and the genotype is obtained from. The training set is the reference transcriptome dataset where the prediction models of gene expression levels are trained. The reference set, in general 1000 Genomes, is used to compute the variances and covariances (LD structure) of the markers used in the predicted expression levels. Both the reference set and training set values are pre-computed and provided to the user so that only the study set results need to be provided to the software. The crossed out term was set to $1$ as an approximation, since its calculation depends on generally unavailable data. We found this approximation to have negligible impact on the results.} % I like this fig! add something like: The crossed-out term defines the xxxx, but we show that without it, the results do not appreciably change. -HEW
\label{fig:metaxcan-formula}
\end{center}
\end{figure}

The approximate formula we will use is as follows:
\begin{equation}
Zg \approx \sum_{l\in \text{Model}_g} w_{lg} ~\frac{\hat\sigma_l}{\hat\sigma_g} ~  \frac{\hat\beta_l}{\text{se}(\beta_l)} 
% =======
% Zg = \sum_{l\in \text{Model}_g} w_{lg} \frac{\hat\beta_l}{\text{se}(\hat\beta_l)} \frac{\sigma_l}{\hat\sigma_g}
% >>>>>>> 6fb1a98ff5949785f2a2d8324997441ea0c84a46
\end{equation}

where

\begin{itemize}
\item $w_{lg}$ is the weight of SNP $l$ in the prediction of the expression of gene $g$,
\item $\hat\beta_l$ is the GWAS regression coefficients for SNP $l$,
\item se($\beta_l$) is standard error of $\hat\beta_l$,
\item $\hat\sigma_l$ is the estimated variance of SNP $l$, and
\item $\hat\sigma_g$ is the estimated variance of the predicted expression of gene $g$.
\end{itemize}

The inputs are based, in general, on data from three different sources: 
\begin{itemize}
\item study set,
\item training set,
\item population reference set. 
\end{itemize}

The study set is the main dataset of interest from which the genotype and phenotypes of interest are gathered. 
% I am not sure about this sentence. - AB
The regression coefficients and standard errors are computed based on individual-level data from the study set. Training sets are the reference transcriptome datasets used for the training of the prediction models (GTEx, DGN, Framingham, etc.) thus the weights $w_{lg}$ are computed from this set. Finally, the reference sets (e.g. 1000 Genomes) are used to derive variance and covariance (LD) properties of genetic markers, which will usually be different from the study sets.

In the most common use scenario, the user will only need to provide GWAS results using his/her study set. The remaining parameters are pre-computed, and download information can be found at the \url{https://github.com/hakyimlab/MetaXcan} resource.

Next we will show the performance of the method, measured as the concordance (R$^2$) between PrediXcan and MetaXcan results.


\subsubsection*{Performance in simulated data}

We first compared MetaXcan and PrediXcan using simulated phenotypes generated from a normal distribution, using a single transcriptome model trained on Depression Genes and Network's (DGN) Whole Blood data set \cite{Battle2014} downloaded from PredictDB (\url{http://predictdb.org}). As genotypes we used three ancestral subsets of the 1000 Genomes project: Africans (n=662), East Asians (n=504), and Europeans (n=503). Each set was taken in turn as reference and study set yielding a total of 9 combinations as shown in Figure \ref{fig:simulatedgrid}. For each population combination, we computed PrediXcan association results for the simulated phenotype and compared them with results generated from our MetaXcan approach in a scatter plot. This allowed us to assess the effect of ancestral differences between study and reference sets.

As expected, when the study and reference sets are the same, the concordance between MetaXcan and PrediXcan is 100\% whereas for sets of different ancestral origin the R$^2$ drops a few percentage points, with the biggest loss (down to 85\%) when the study set is African and the reference set is Asian. This confirmed that our formula works as expected and that the approach is robust to ethnic differences between study and reference sets.

\begin{figure}
\includegraphics[width=0.6\textwidth]{plots/Fig3-simulated_grid.png}
\caption{Comparison of PrediXcan and MetaXcan results for a simulated phenotype. 
Study populations and MetaXcan reference populations were built from European, African, and Asian
individuals from the 1000 Genomes Project. Gene Expression model was based on DGN's Whole Blood data.
%Dot color accounts for number of SNPS in each gene.\hl{TODO (WTCCC)}
}
\label{fig:simulatedgrid}
\end{figure}

\subsubsection*{Performance in cellular growth phenotype from 1000 genomes cell lines}
 
Next we tested with an actual cellular phenotype. Intrinsic growth, a cellular phenotype, was computed based on multiple growth asays for over 500 cell lines from the 1000 Genomes project \cite{Im2012}. We used a subset of values for Europeans (EUR), Africans (AFR), Asians (EAS) individuals.
 
%-Assuming ASW is included in the African set, I'd write it like this. -HEW 
% Values in parentheses are verbatim from 1000 genomes data. Reverted text to 1000 Genomes tag. -AB
% * <kaananshah@gmail.com> 2015-11-10T19:57:32.009Z:
%
% > African American (\hl{ASW?})
%
% Are these in fig 4?
%
% ^.

We compared Z-scores for  intrinsic growth generated by PrediXcan and MetaXcan for different combinations of reference and study sets, using whole blood prediction model trained in the DGN cohort. The results are shown in Figure \ref{fig:igrowthgrid}. Consistent with our simulation study, the MetaXcan results closely match the PrediXcan results. Again, the best concordance occurs when reference and study sets share similar continental ancestry while differences in population slightly reduce concordance. Compared to the plots for the simulated phenotypes, the diagonal concordance is slightly lower than 1. This is due to the fact that more individuals were included in the reference set than in the study set, thus the study and reference sets were not identical for MetaXcan.

\begin{figure}
\includegraphics[width=0.6\textwidth]{plots/Fig4-igrowth_grid.png}
\caption{Comparison of PrediXcan and MetaXcan results for a cellular phenotype, intrinsic growth. 
Study sets and MetaXcan reference sets consisted of European, African, and Asian
individuals from the 1000 Genomes Project. Gene Expression model was based on Depression Genes and Networks.}
\label{fig:igrowthgrid}
\end{figure}

\subsubsection*{Performance on disease phenotypes from WTCCC}

We show the comparison of MetaXcan and PrediXcan results for two diseases: Bipolar Disorder (BD) and Type 1 Diabetes (T1D) from the WTCCC in Figure \ref{fig:BDT1DMP}. Other disease phenotypes exhibited similar performance (data not shown). Concordance between MetaXcan and PrediXcan is over 95\% in for both diseases (BD $R^2=0.956$ and T1D $R^2=0.958$). The very small discrepancies are explained by differences in allele frequencies and LD between the reference set (1000 Genomes) and the study set (WTCCC). Given this high concordance, we do not expect much improvement when using a reference set that is more similar to the study set. We verified this and, as expected, found that using control individuals from WTCCC as reference set improved the concordance only marginally (0.1\%).

It is worth noting that the PrediXcan results for diseases were obtained using logistic regression whereas MetaXcan formula is based on linear regression properties. As observed before \cite{Zhou2013}, when the number of cases and controls are relatively well balanced (roughly, at least 25\% of cases and controls), linear regression approximation yields very similar results to logistic regression.

This high concordance also shows that the approximation where we drop the term $\sqrt{\frac{1-R_l^2}{1-R_g^2}}$ does not significantly affect the results.

\begin{figure}
\includegraphics[width=\textwidth]{plots/Fig5-BDT1D.png}
\caption{Comparison of PrediXcan results and MetaXcan results for a Type I Diabetes study, and a Bipolar Disorder study.
Study data was extracted from Wellcome Trust Case Control Consortium,
and MetaXcan reference population were the European individuals from Thousand Genomes Project
(same as in previous sections) }
-\label{fig:BDT1DMP}
\end{figure}

% \subsubsection*{Application to large-scale meta analysis results}

% Finally, we apply MetaXcan to a number of publicly available large consortia meta analysis results with sample sizes exceeding the 100K individuals. The full list of GWAS studies used are listed on table \ref{tab:consortia}. Using gene expression prediction models for 42 different tissue models, we performed MetaXcan association. As expected, we find that the top hits are enriched for known disease/trait associated genes. We make the results available through \url{gene2pheno.org}, which should be useful to the community for replication, validation, or as functional annotations of genes. Disease/trait focused analysis of these results are currently ongoing to fully leverage these results and further our biological understanding of the traits.
% % * <kaananshah@gmail.com> 2015-11-10T20:08:05.683Z:
% %
% % > 39 different tissues
% %
% % is it 39 or 40? the abstract says 40
% %
% % ^.
% % - The end version of this used 42 (40GTEX + Crosstissue + DGNWB)




\subsubsection*{Software}

We make our software publicly available on a GitHub repository: \url{https://github.com/hakyimlab/MetaXcan}. Instructions for obtaining the weights and covariances for different tissues can be found there. A short working example can be found on the GitHub page; more extensive documentation can be found on the project's wiki page.

% For ease of use, we have developed a desktop GUI application. We have also developed a web-based version to further simplify the use of the method, available at: \url{http://hakyimlab.org/metaxcan/})

%\begin{figure}
%\includegraphics[width=0.6\textwidth]{plots/gui.png}
%\caption{ GUI application running in a local environment. }
%\label{fig:gui}
%\end{figure}
 
%\begin{figure}
%\includegraphics[width=0.6\textwidth]{plots/WebAppExample.png}
%\caption{ Web Based version, available at \url{http://hakyimlab.org/metaxcan/}}
%\label{fig:webapp}
%\end{figure}

\section*{Discussion}

Here we present MetaXcan, a scalable, accurate, and efficient method for integrating reference transcriptome studies to learn about the biology of complex traits and diseases. Our method extends PrediXcan, which maps genes to phenotypes by testing the mediating effects of gene expression levels. This is implemented by predicting gene expression levels  and correlating these traits with phenotypes. MetaXcan is a shortcut that uses SNP-level association results and combines them to reproduce the results of PrediXcan, without the need to use individual level data.

MetaXcan shares most of the benefits of PrediXcan: a) it directly tests the regulatory mechanism through which genetic variants affect phenotype; b) it provides gene-level results which are better functionally characterized than genetic variants, easier to validate within model systems, and carry a smaller multiple testing burden; c) the direction of the effects are known, facilitating identification of therapeutic targets; d) reverse causality is largely avoided since predicted expression levels are based on germline variation, which are not affected by onset of disease; e) it can be systematically applied to existing GWAS studies; f) tissue-specific analysis can be performed using all the models we have made available through PredictDB (\url{http://predictdb.org}).
% * <kaananshah@gmail.com> 2015-11-10T20:14:14.597Z:
%
% > over 40
%
% what is the actual number of tissue models we have?
%
% ^.

The difference between the reference sets (used to estimate LD and allele frequencies) and study set (used to compute GWAS/meta analysis summary statistics) is the main cause of the small differences between MetaXcan and PrediXcan results. We have shown here that even when the populations are quite different, the concordance is very high. Thus, MetaXcan is robust to ancestral differences between study and reference sets.

Even though the method was derived with linear regression in mind, in case-control designs, the approximation generates results that are in almost full  concordance with exact results generated with PrediXcan and logistic regression.

Methods similar in spirit to PrediXcan have been reported \cite{Gusev2016}. Gusev et al also propose a method comparable to MetaXcan that is based only on summary statistics. Their method, called Transcriptome-Wide Association Study (TWAS), imputes the SNP level z-scores into gene level z-scores using the method Pasaniuc and others have published \cite{Pasaniuc2014}. This approach is equivalent to predicting expression levels using BLUP/Ridge Regression, which has been shown to be suboptimal for prediction. This is due to the fact that the local architecture of gene expression traits is sparse so that highly polygenic models underperform more sparse prediction models such as LASSO or Elastic Net with mixing parameters 0.5 or greater \cite{Wheeler2016}.

In contrast, MetaXcan is not restricted to one imputation or prediction scheme. It
infer the results of PrediXcan using summary statistics through an analytic formula. Thus it can be applied to linear models based on SNP data.

% * <kaananshah@gmail.com> 2015-11-10T20:16:52.372Z:
%
% > imputation scheme
%
% might be worth clarifying here that our current models are based on an elastic net with alpha = 0.5, and the TWAS method is effectively using a ridge regression approach (i think... based on haky's comments). We showed in the first paper that EN was better than RR and therefore expect our method to work better? 
%
% ^. --I agree with Kaanan -HEW
% The previous sentence is unclear, analytic formula? -HEW

In summary, we present an accurate and computationally efficient gene-level association method that integrates functional information from reference transcriptome dataset into GWAS and large scale meta-analysis results to inform the biology of complex traits.
% * <kaananshah@gmail.com> 2015-11-10T20:18:28.733Z:
%
% > computationally efficient
%
% since we repeatedly state that this method is efficinet, should we have some numbers on commute time? even a comparison of Predixcan from start to finish vs metaxcan in the WTCCC would be good to show? Especially if you do not include the time it takes to generate the GWAS summary stats (which is reasonable since people would be applying this as a secondary analysis)...
%
% ^.


%It not not hard to show that the imputation scheme in Guset et al amounts to using ridge regression prediction of gene expression traits. Our study of the genetic architecture of gene expression traits have shown that ridge regression performs worse than more sparse models such as elastic net or lasso.
%
%From what we are finding about the genetic architecture of gene expression traits, we have evidence that indicates that the imputation scheme used in TWAS (the summary statistics one) is sub-optimal \cite{}. Gusev et al's summary stat method is equivalent to kriging the Zscores, thus the performance is tied to the predictive performance of kriging gene expression traits. Notice that kriging and ridge regression are equivalent in terms of prediction. Since we are finding that ridge regression (mixing parameter = 1 in Elastic net) has lower performance consistently across all genes, we can expect that the imputation of Zscore approach will perform worse than MetaXcan, if we use elastic net weights, for example.



\section*{Methods}

\subsection*{Derivation of MetaXcan Formula}

The goal of MetaXcan is to infer the results of PrediXcan using only GWAS summary statistics. Individual level data are not needed for this algorithm. We will define some notations for the derivation of the analytic expressions of MetaXcan.

\subsubsection*{Notation and Preliminaries}

$Y$ is the $n$-dimensional vector of phenotype for individuals $i=1,n$.\\
$X_l$ is the allelic dosage for SNP $l$.\\
$T_g$ is the predicted expression (or estimated GREx, genetically regulated expression).\\
We model the phenotype as linear functions of $X_l$ and $T_g$
\begin{align*}
Y &= X_l \beta_l + \eta\\
Y &= T_g \gamma_g + \epsilon,
\end{align*}
where $\hat\gamma_g$ and $\hat\beta_l$ are the estimated regression coefficients of $Y$ regressed on $T_g$ and $X_l$, respectively. $\hat\gamma_g$ is the result (effect size for gene $g$) we get from PrediXcan whereas $\hat\beta_l$ is the result from a GWAS for SNP $l$.

We will denote as Var and Cov the operators that computes the sample variance and covariances, i.e. Var($Y$) = $\sum_{i=1,n} (Y_i - \bar{Y})^2/n$ with $\bar{Y} = \sum_{i=1,n} Y_i / n$\\
$\hat\sigma^2_l = $  Var$(X_l)$\\
$\hat\sigma^2_g = $  Var$(T_g)$\\
$\hat\sigma^2_Y = $  Var$(Y)$\\
$\Gamma_g = \mathbf{(X-\bar{X})'(X-\bar{X})}/n$,\\ 
 where  $\mathbf{X'}$ is the $n \times p$ matrix of SNP data and $\mathbf{\bar{X}}$ is a $n \times p$ matrix where column $l$ has the column mean of $\mathbf{X}_l$ ($p$ being the number of SNPS in the  model for gene $g$).\\
% * <jason.matthew.torres@gmail.com> 2015-11-11T17:58:12.506Z:
%
% Can you make it more explicit, where the g fits into the covariance formula. I'm not sure if readers will readily appreciate that this refers to the convariance of SNP predictors for gene g. 
%
% ^.

With this notation, our goal is to infer PrediXcan results ($\hat\gamma_g$ and its standard error) using only GWAS results ($\beta_l$ and se), estimated variances of SNPs 
($\hat\sigma^2_l$), covariances between SNPs in each gene model ($\Gamma_g$), and prediction model weights $w_{lg}$.\\

\textbf{Input:}  $\beta_l$, se($\beta_l$), $\hat\sigma^2_l$, $\Gamma_g$, $w_{lg}$.
\textbf{Output:} 
$\hat\gamma_g$, se($\hat\gamma_g$).\\
%From these the Zscore ($Z = \hat\gamma_g$/ se($\hat\gamma_g$)) and p value ($p$= 2 pnorm($-|Z|$) ) can be computed.

Next we list the properties and definitions used in the derivation: 
\begin{equation}
\hat\gamma_g = \frac{ \text{Cov}(T_g,Y) }{ \text{Var}(T_g)}  =  \frac{ \text{Cov}(T_g,Y) }{ \hat\sigma^2_g }
\end{equation}
and  
\begin{equation}
\hat \beta_l = \frac{ \text{Cov}(X_l,Y) }{ \text{Var}(X_l) } =  \frac{ \text{Cov}(X_l,Y) }{ \hat\sigma_l^2 } \label{eq.beta}
\end{equation}
The proportion of variance explained by the covariate ($T_g$ or $X_l$) can be expressed as\\
\begin{equation}
R_g^2 = \hat\gamma^2_g ~ \frac{\hat\sigma_g^2 }{ \hat\sigma^2_Y} \nonumber
\end{equation}
\begin{equation}
R_l^2 = \hat\gamma^2_l ~ \frac{\hat\sigma_l^2 }{ \hat\sigma^2_Y} \nonumber
\end{equation}
By definition
\begin{equation}
T_g = \sum_{l \in \text{Model}_g} w_{lg}X_l 
\end{equation}

Var($T_g$) = $\hat\sigma^2_g$ can be computed as
\begin{align}
\hat\sigma^2_g & = \text{Var} \left ( \sum_{l \in \text{Model}_g} w_{lg} X_l \right )\nonumber  & \text{}\\  
& = \text{Var} (\mathbf{W}_g \mathbf{X}_g) \nonumber  & \text{where } \mathbf{W}_g \text{is the vector of } w_{lg} \text{for SNPs in the model of } g \\
& = \mathbf{W}_g' \text{Var}( \mathbf{X}_g )\mathbf{W}_g \nonumber  & \text{where } \Gamma_g \text{ is the} \text{Var}( \mathbf{X}_g ) = \text{ covariance matrix of } \mathbf{X}_g \\
& = \mathbf{W}_g'  \Gamma_g \mathbf{W}_g & \text{} \label{eq.var.g}
\end{align}

\subsubsection*{Calculation of regression coefficient $\gamma_g$}

$\hat\gamma_g$ can be expressed as
\begin{align}
\hat \gamma_g & = \frac{ \text{Cov}(T_g,Y) }{ \hat\sigma^2_g } & \nonumber \\
& = \frac{ \text{Cov}(\sum_{l \in \text{Model}_g} w_{lg}X_l,Y) }{ \hat\sigma^2_g } & \nonumber \\
& = \sum_{l\in \text{Model}_g} \frac{w_{lg} \text{Cov}(X_l, Y) }{ \hat\sigma^2_g } & \text{by linearity of Cov} \nonumber \\
& = \sum_{l\in \text{Model}_g} \frac{ w_{lg}\hat\beta_l \sigma^2_l }{ \hat\sigma^2_g } & \text{using Eq \ref{eq.beta}} \label{eq.gamma}
\end{align}

\subsubsection*{Calculation of standard error of $\gamma_g$}
Also from the properties of linear regression we know that 
\begin{equation}
\text{se}(\hat\gamma_g) = \sqrt{\text{Var}(\hat\gamma_g)} = \frac{\hat\sigma_\epsilon}{ \sqrt{n \hat\sigma^2_g} } = \frac{\hat\sigma^2_Y (1-R^2_g)}{n\hat\sigma_g^2} \label{eq.se.gamma}
\end{equation}
% where
% \begin{equation}
% \hat\sigma_\epsilon^2 = \sigma_Y^2 (1 - R^2) = \hat\sigma_Y^2 (1 - \hat\sigma^2_g \hat\gamma_g^2 / \hat\sigma_Y^2) =
% \hat\sigma_Y^2 - \hat\sigma^2_g \hat\gamma_g^2  \label{eq.sigma.epsilon}
% \end{equation}
% Thus
% \begin{equation}
% \text{se}(\hat\gamma_g)^2 = \frac{\hat\sigma_Y^2 - \hat\sigma^2_g \hat\gamma_g^2  }{ n \hat\sigma^2_g } \label{eq.se.gamma}
% \end{equation}
In this equation, $\sigma_Y/n$ is not necessarily known but can be estimated using the analogous equation (\ref{eq.se.gamma}) for beta
\begin{equation}
\text{se}(\hat\beta_l) = \frac{\hat\sigma^2_Y (1-R^2_l)}{n\hat\sigma_l^2} \label{eq.se.beta}
\end{equation}
Thus
\begin{equation}
\frac{\hat\sigma_Y^2}{n} = \frac{\text{se}(\hat\beta_l)^2 \hat\sigma_l^2}{(1-R^2_l)} \label{eq.sigYn}
% \hat\sigma_Y^2 &= \text{se}(\hat\beta_l)^2  n \hat\sigma^2_l + \hat\sigma^2_l \hat\beta_l^2   \\
% \hat\sigma_Y^2 &= \hat\sigma^2_l ( n \text{ se}(\hat\beta_l)^2  +  \hat\beta_l^2  )
\end{equation}

Notice that the right hand side of (\ref{eq.sigYn}) is dependent on the SNP $l$ while the left hand side is not. This equality will hold only approximately in our implementation since we will be using approximate values for $\hat\sigma_l^2$, i.e. from reference population, not the actual study population.

\subsubsection*{Calculation of Z score}

To assess the significance of the association, we need to compute the ratio of the effect size $\gamma_g$ and standard error se($\gamma_g$), or Z score, 
\begin{align}
Z_g & = \frac{ \hat\gamma_g}{\text{se}(\hat\gamma_g)} 
\end{align}

with which we can compute the  p value as
\begin{equation}
p = 2 \text{ pnorm}(-|Z_g|)
\end{equation}

\begin{align}
Zg 
& = \frac{ \hat\gamma_g}{\text{se}(\hat\gamma_g)} \nonumber 
&\text{}\\
%
& =   \sum_{l\in \text{Model}_g} \frac{w_{lg} \hat\beta_l \sigma^2_l }{ \hat\sigma^2_g}   \sqrt{\frac{n}{\hat\sigma_Y^2 }\frac{\hat\sigma^2_g}{(1-R^2_g)} }  
&\text{using Eq. \ref{eq.gamma} and \ref{eq.se.gamma}} \nonumber \\ 
%
&= \sum_{l\in \text{Model}_g}\frac{  w_{lg} \hat\beta_l \sigma^2_l }{ \hat\sigma_g}  \sqrt{\frac{(1-R^2_l)}{\text{se}(\hat\beta_l)^2 \hat\sigma_l^2}} \sqrt{\frac{1}{(1-R^2_g)} } 
& \text{} \nonumber \\
%
&= \sum_{l\in \text{Model}_g}  w_{lg} ~ \frac{\sigma_l}{\hat\sigma_g} ~ \frac{\hat\beta_l}{\text{se}(\hat\beta_l)} ~ \sqrt{\frac{1-R_l^2}{1-R_g^2}}
& \text{}\\
%
&\approx \sum_{l\in \text{Model}_g}  w_{lg} ~ \frac{\sigma_l}{\hat\sigma_g} ~ \frac{\hat\beta_l}{\text{se}(\hat\beta_l)} 
& \text{}
\end{align}

%for latexit slide

% $ Z_g = \sum_{l\in \text{Model}_g}  w_{lg} ~ \frac{\sigma_l}{\mathbf{W}_g'  \Gamma_g \mathbf{W}_g} ~ \frac{\hat\beta_l}{\text{se}(\hat\beta_l)} ~ \sqrt{\frac{1-R_l^2}{1-R_g^2}} $

% Z_g = \sum_{l\in \text{Model}_g}  w_{lg} ~ \frac{\sigma_l}{\hat\sigma_g} ~ \frac{\hat\beta_l}{\text{se}(\hat\beta_l)} ~ \sqrt{\frac{1-R_l^2}{1-R_g^2}}

%\hat \gamma_g & = \frac{ \text{Cov}(T_g,Y) }{ \hat\sigma^2_g } & \nonumber \\
%& = \frac{ \text{Cov}(\sum_{l \in \text{Model}_g} w_{lg}X_l,Y) }{ \hat\sigma^2_g } & \nonumber \\
%& = \sum_{l\in \text{Model}_g} \frac{w_{lg} \text{Cov}(X_l, Y) }{ \hat\sigma^2_g } & \text{} \nonumber \\
%& = \sum_{l\in \text{Model}_g} \frac{ w_{lg}\hat\beta_l \sigma^2_l }{ \hat\sigma^2_g } & \text{} 



Based on results with actual and simulated data we have found that the last approximation does reduce power since the deviation is only noticeable when the correlation between the SNP or the predicted expression and the phenotype is large, i.e. large effect sizes. When the effects are large the loss of power is compensated by the large effect size.
% * <kaananshah@gmail.com> 2015-11-10T20:24:06.965Z:
%
% > When the effects are large the affect the approximation, the loss of power is compensated by the large effect size.
%
% i dont get this sentence.. reword? 
%
% ^.
% - I don't get it either - AB



%
%\section*{Supplementary Material}
%
%
%The p-values of MetaXcan results were compared against the uniform distribution,
%and we display the output in Figure \ref{fig:qqsimulatedgrid}, in logarithmic scale,
%for different population combinations.
%As expected, the p-values are uniformly distributed when using ethnically related
%population sets for the Study and Summary Statistics Reference populations.
%Using a study population that differs from the reference population
%yielded less precise results.
%
%\begin{figure}
%\includegraphics[width=0.6\textwidth]{plots/qq_simulated_grid.png}
%\caption{Q-Q plot of MetaXcan P-Values assuming uniform distribution, for the different combinations of Study and Reference Population from Figure \ref{fig:simulatedgrid}.}
%\label{fig:qqsimulatedgrid}
%\end{figure}
%
%Figure \ref{fig:qqigrowthgrid} shows the Q-Q plot of the p-values for this analysis. When the same study population is used
%in the summary statistics, no genes appear as significantly relevant to the phenotype, according to the transcriptome model.
%
%
%\begin{figure}
%\includegraphics[width=0.6\textwidth]{plots/qq_igrowth_grid.png}
%\caption{Q-Q plot of MetaXcan P-Values assuming uniform distribution, for the different combinations of Study and Reference Population from Figure \ref{fig:igrowthgrid}.}
%\label{fig:qqigrowthgrid}
%\end{figure}
%

% Do NOT remove this, even if you are not including acknowledgments
\section*{Acknowledgments}

\subsection*{Grants}\label{grants}

We acknowledge the following US National Institutes of Health grants:
R01MH107666 (H.K.I.), K12 CA139160 (H.K.I.), T32 MH020065 (K.P.S.), R01 MH101820 (GTEx), 
P30 DK20595 and P60 DK20595 (Diabetes Research and
Training Center), P50 DA037844 (Rat Genomics), P50 MH094267 (Conte). H.E.W. was
supported in part by start-up funds from Loyola University Chicago.

% this dataset is not used for this paper so excluded ack. Only dgn...db downloaded from PredictDB.
% \paragraph{DGN data.}\label{dgn-data}
% NIMH Study 88 -- Data was provided by
% Dr.~Douglas F. Levinson. We gratefully acknowledge the resources were
% supported by National Institutes of Health/National Institute of Mental
% Health grants 5RC2MH089916 (PI: Douglas F. Levinson, M.D.;
% Coinvestigators: Myrna M. Weissman, Ph.D., James B. Potash, M.D., MPH,
% Daphne Koller, Ph.D., and Alexander E. Urban, Ph.D.) and 3R01MH090941
% (Co-investigator: Daphne Koller, Ph.D.).

%Data on birth weight trait has been contributed by EGG Consortium and has been downloaded from www.egg-consortium.org.

%Data on glycaemic traits have been contributed by MAGIC investigators and have been downloaded from www.magicinvestigators.org

%\section*{References}

%%\begin{table}[!ht]
%%\caption{
%%\bf{Table title}}
%%\begin{tabular}{|c|c|c|}
%%table information
%%\end{tabular}
%%\begin{flushleft}Table caption
%%\end{flushleft}
%%\label{tab:label}
%% \end{table}

\bibliography{metaxcan-derivation}

% \section*{Tables}
% \begin{table}
%   \begin{tabular}{| r | r |} 
%     \hline
% 	\rowstyle{\bfseries} Consortium & \rowstyle{\bfseries} Study \\
%     \hline
%     AMD & Advanced vs Control \cite{Fritsche2013}\\ \hline
%      & Geographic Atropy vs Control \cite{Fritsche2013} \\ \hline
%      & Neovascular Disease vs Control \cite{Fritsche2013} \\ \hline
%     EGG & Birth Weight summary Data \cite{Horikoshi2012} \\ \hline
%     GEFOS & Femoral Neck BMD (Pooled) \cite{Estrada2012} \\ \hline
%      & Lower Spine BMD (Pooled) \cite{Estrada2012} \\ \hline
%     GIANT & Anthropometric 2015 BMI EUR \cite{Locke2015} \\ \hline
%      & HDL Cholesterol \cite{Locke2015} \\ \hline
%      & LDL Cholesterol \cite{Locke2015} \\ \hline
%      & Total Cholesterol \cite{Locke2015} \\ \hline
%      & Triglicerids \cite{Locke2015} \\ \hline
%     IBD & European CD \cite{Liu2015} \\ \hline
%      & European IBD \cite{Liu2015} \\ \hline
%      & European UC \cite{Liu2015} \\ \hline
%     MAGIC & 2hr Glucose Adjusted For BMI \cite{Saxena2010} \\ \hline
%      & Fasting Glucose \cite{Dupuis2010} \\ \hline
%      & Fasting Insulin \cite{Dupuis2010} \\ \hline
%      & Fasting Proinsulin \cite{Strawbridge2011} \\ \hline
%      & HOMA-B \cite{Dupuis2010} \\ \hline
%      & HOMA-IR \cite{Dupuis2010} \\ \hline
% 	 & HbA1C \cite{Soranzo2010} \\ \hline
%      & Fasting Glucose Interaction \cite{Manning2012} \\ \hline
%      & Fasting Glucose Main Effect \cite{Manning2012} \\ \hline
%      & Fasting Insulin Interaction \cite{Manning2012} \\ \hline
%      & Fasting Insulin Main Effect \cite{Manning2012} \\ \hline
%     PGC & ADHD \cite{Neale2010} \\ \hline
%      & AUT Eur \cite{Smoller2013} \\ \hline
%      & Cross Disorder Full \cite{Smoller2013} \\ \hline
%      & Cross Disorder ADD Subset \cite{Smoller2013} \\ \hline
%      & Cross Disorder AUT Subset \cite{Smoller2013} \\ \hline
%      & Cross Disorder BIP Subset \cite{Smoller2013} \\ \hline
%      & Cross Disorder MDD Subset \cite{Smoller2013} \\ \hline
%      & Cross Disorder SCZ Subset \cite{Smoller2013} \\ \hline
%      & MDD clump \cite{Ripke2013} \\ \hline
%      & MDD Full \cite{Ripke2013} \\ \hline
%      & SCZ \cite{RipkeSSanders2011} \\ \hline
%     RACI GARNET & Rheumatoid Arthritis, Asian \cite{Okada2014} \\ \hline
%      & Rheumatoid Arthritis, European \cite{Okada2014} \\ \hline
%      & Rheumatoid Arthritis, TransEthnic \cite{Okada2014} \\ \hline
%     \end{tabular}
%   \caption{List of GWAS studies analyzed with MetaXcan}
%   \label{tab:consortia}
% \end{table}



\end{document}


